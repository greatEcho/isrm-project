%% Преамбула TeX-файла

% 1. Стиль и язык
\documentclass[utf8x]{G7-32} % Стиль (по умолчанию будет 14pt)
\usepackage[T2A]{fontenc}
\usepackage[russian]{babel}
% Остальные стандартные настройки убраны в preamble.inc.tex.
% \include{preamble.inc}

% Настройки листингов.
% \include{listings.inc}

% Полезные макросы листингов.
% \include{macros.inc}

\makeatletter
\input{preamble.inc}
\input{listings.inc}
\input{macros.inc}
\makeatother

\begin{document}

\frontmatter % выключает нумерацию ВСЕГО; здесь начинаются ненумерованные главы: реферат, введение, глоссарий, сокращения и прочее.

% Команды \breakingbeforechapters и \nonbreakingbeforechapters
% управляют разрывом страницы перед главами.
% По-умолчанию страница разрывается.

% \nobreakingbeforechapters
% \breakingbeforechapters

\singlespacing
\include{00-titlepage}
% \include{01-executors}
\onehalfspacing
% \include{02-abstract}

\pdfbookmark[1]{СОДЕРЖАНИЕ}{TOC}
\tableofcontents

\include{03-defines}
\include{04-abbrev}

% Введение
\include{05-intro}

\mainmatter % это включает нумерацию глав и секций в документе ниже
\include{work}
% \input{10-aao_section1}
% \input{17-polar_codes}
% \input{18-LDPC}
% \include{20-coding_for_4g}
% \include{30-5G_polar}
% \include{31-5G_LDPC}
% input чтобы несоздавался разрыв страницы между файлами
% \input{41-LTE_Matlab}
% \input{42-framework}
% \input{43-complexity}
% \include{50-simulation}

\backmatter %% Здесь заканчивается нумерованная часть документа и начинаются ссылки и
            %% заключение

\include{80-conclusion}

\include{81-biblio}

% \appendix   % Тут идут приложения

% \include{91-appendix-complex-table}
% \include{92-appendix-FER}
% \include{93-appendix-complexity}

\end{document}

%%% Local Variables:
%%% mode: latex
%%% TeX-master: t
%%% End:
